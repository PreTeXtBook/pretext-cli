% Sage CLI Quick Reference
% (c) 2022 by Steven Clontz, Thomas W.\ Judson, and Oscar Levin
% Licensed with the GNU Free Documentation License (GFDL)
%   http://www.gnu.org/copyleft/fdl.html
%
%  History
%
%    2012-06-15  Initial version based on Sage 9.4
%
%
\documentclass{article}
\usepackage{graphicx}  
\usepackage[landscape]{geometry}
\usepackage[pdftex]{color}
\usepackage{url}
\usepackage{multicol}
\usepackage{amsmath}
\usepackage{amsfonts}
\usepackage{textcomp}
\newcommand{\ex}{\color{blue}}
\newcommand{\apost}{\textquotesingle}
\newcommand{\warn}{\bf\color{red}}
\pagestyle{empty}
\advance\topmargin-.9in
\advance\textheight2in
\advance\textwidth3.0in
\advance\oddsidemargin-1.45in
\advance\evensidemargin-1.45in
\parindent0pt
\parskip2pt
% Section break, dictates column widths?
\newcommand{\hr}{\centerline{\rule{3.5in}{1pt}}}
% Adjust gap to affect spacing, page count
\newcommand{\sect}[1]{\hr\par\vspace*{2pt}\textbf{#1}\par}
% Mandatory indentation on subsidiary lines
\newcommand{\skipin}{\hspace*{12pt}}
% notation shortcut
\newcommand{\Z}{\mathbb{Z}}
\begin{document}
\begin{multicols*}{3}
\begin{center}
\textbf{PreTeXt Quick Reference: Command Line Interface (CLI)}\\
% Steve Clontz, T.\ W.\ Judson, and Oscar Levin \\
CLI version 0.8.0, 6/24/2022  \\
% Switch to CC?
GNU Free Document License, extend for your own use
\end{center}
% backup over center environment gap
\vspace{-2ex}
%*********************************************
\sect{Setup}
\hr\textbf{Check requirements}\\
\underline{Note}: {\verb|python|} might be called {\verb|python3|} if you have MacOS or Linux.

{\ex \verb| python --version|} : the CLI requires Python version 3.8 or later\\
{\ex \verb| pip --version|} : pip is the package installer for Python \\
{\ex \verb| xelatex --version|} : some PreTeXt features require \LaTeX \\

\hr\textbf{Install PreTeXt}\\
{\ex \verb| python -m pip install pretext-cli|} : install PreTeXt \\
{\ex \verb| pretext --version|} : check version to verify install \\



\hr\textbf{Create a new project}\\
{\ex \verb| pretext new book|} : creates a new PreTeXt book in \texttt{new-pretext-project}\\
{\ex \verb| pretext new article|} : creates a new PreTeXt article \texttt{new-pretext-project}

\hr\textbf{Update a project to use the CLI}\\
{\ex \verb| pretext init|}: creates project manifest ({\verb|project.ptx|}), and publication file ({\verb|publication/publication.ptx|}).  Edit these files appropriately before proceeding. \\
{\ex \verb| pretext pretext init --refresh|}: creates new copies of project manifest and publication file to compare for new features.

\hr\textbf{Upgrade PreTeXt}\\
{\ex \verb| python -m pip install --upgrade pretext-cli|}: upgrade to latest stable release \\

\hr\textbf{Get Help}\\
{\ex \verb| pretext --help|}: show general help\\
{\ex \verb| pretext build --help|}: show help for build command.  Each subcommand has its own help.

%
%*********************************************
%*********************************************
\columnbreak
\sect{Basic Usage}
 \hr\textbf{Build a PreTeXt document}

{\ex \verb| pretext build|}: Builds the project to the format of the first target in {\verb|project.ptx|}.\\
{\ex \verb| pretext build web|}: Create html version (assuming {\verb|<target name="web">|}) is in {\verb|publication.ptx|}\\
{\ex \verb| pretext build print|}: Create print (pdf) version\\


\hr\textbf{Generate source images and WeBWorK}\\
If your book has any WeBWorK, latex-image, asymptote, sageplot, interactive, etc. you need to generate these from source.

{\ex \verb| pretext generate|}: Generate all assets for first target in {\verb|project.ptx|}.\\
{\ex \verb| pretext generate webwork|}: Generate webwork for first target in {\verb|publication.ptx|}\\
{\ex \verb| pretext generate sageplot -t print|}: Generate sageplot for target "print".\\
{\ex \verb| pretext generate latex-image -x img-graph1|}: Generate latex-image with xml:id "img-graph1" (for first target).\\
%*********************************************

 \hr\textbf{View a PreTeXt document (local)}

{\ex \verb| pretext view|}: Creates a local server to preview the first target in {\verb|project.ptx|}\\
{\ex \verb| pretext view print|}: Views the "print" target\\
{\ex \verb| pretext view -w|}: Builds, starts server, and rebuilds every time the project is saved\\
{\ex \verb| CTRL+C|} to close the server\\
%*********************************************


%*********************************************
\hr\textbf{Deploy to GitHub Pages}

{\ex \verb| pretext deploy|} : deploys Git-managed project to GitHub Pages\\
{\ex \verb| pretext deploy -u|} : deploys and also uploads source files\\


%*********************************************



\columnbreak
%*********************************************

\sect{Useful Shortcuts}

{\ex \verb| pretext build -g|}: build and generate in one step\\
{\ex \verb| pretext build web -g latex-image|}: build web target and generate latex-images\\
{\ex \verb| pretext view -b|}: build before you preview\\
{\ex \verb| pretext view -g|}: generate assets before you view\\
{\ex \verb| pretext view -bg|}: build and generate assets before you view\\


%*********************************************
\hr\textbf{Project Manifest}
The file {\verb|project.ptx|} describes your build targets.  Each target has a \emph{name} (e.g. "print-latex") that you build or view with, e.g. {\verb|pretext build print-latex|} or generate assets for with, e.g. {\verb|pretext generate webwork -t print-latex|}.  

Structure of a target:
\begin{verbatim}
  <target name="web">
    <format>html</format>
    <source>source/main.ptx</source>
    <publication>publication/publication.ptx</publication>
    <output-dir>output/web</output-dir>
  </target>
\end{verbatim}
{\ex \verb|<format>|} can be html, latex, or pdf\\
{\ex \verb|<source>|} is the path to the root ptx document\\
{\ex \verb|<publication>|} is the path to the publication file\\
{\ex \verb|<output-dir>|} is the path the the folder that will hold output\\
Other optional elements:\\
{\ex \verb|<stringparam key="..." value="..."/>|}: allows for setting the value of string parameters\\
{\ex \verb|<xmlid-root>ch-first</xmlid-root>|}: used to restrict build to a subset of the source, starting with the element with xml:id "ch-first"\\ 
{\ex \verb|<xsl>xsl/custom-xsl.xsl</xsl>|}: used to build with a custom xsl file.  In that file, import the standard xsl with, e.g.\\ {\verb|<xsl:import pretext-href="pretext-common.xsl"/>|}




\sect{Recommended Project Structure}
{\ex \verb|assets|}: Contains all manually provided assets, such as a photo at {\verb|assets/frog.jpg|} used as {\verb|<image source="frog.jpg"/>|}.\\
{\ex \verb|generated-assets|}: Contains the products of running {\verb|pretext generate|}. Should not be edited manually. \\
{\ex \verb|output|}: Contains the products of running {\verb|pretext build|}. Should not be edited manually. \\
{\ex \verb|publication|}: Contains your publication file(s) (e.g. \verb|publication/publication.ptx|). \\
{\ex \verb|source|}: Contains your PreTeXt source file(s) (e.g. \verb|source/main.ptx|). \\
{\ex \verb|.gitignore|}: Specifies files not shared publicly when using Git or \verb|pretext deploy|. \\
{\ex \verb|project.ptx|}: Describes your project's targets (e.g. \verb|web|, \verb|print-latex|) and executables.\\
{\ex \verb|README.md|}: Written description of your project.\\
{\ex \verb|requirements.txt|}: Specifies version of CLI used to build your project.

%*********************************************
\newpage

\columnbreak

\hr\textbf{publication.ptx}

Information about the publication file goes here.



\columnbreak
\sect{Common PreTeXt source tags}

\hr\textbf{Blocks/Environments}
Example:
\begin{verbatim}
<theorem>
  <title>My Title</title>
  <statement>
    <p>
      Statement of theorem.
    </p>
  </statement>
  <proof>
    <p>
      The proof.
    </p>
  </proof>
</theorem>
\end{verbatim}
Theorem-like: {\ex\verb|<theorem>|}, {\ex\verb|<algorithm>|}, {\ex\verb|<claim>|}, {\ex\verb|<corollary>|}, {\ex\verb|<fact>|}, {\ex\verb|<identity>|}, {\ex\verb|<lemma>|}, {\ex\verb|<proposition>|}.\\
Example-like: {\ex\verb|<example>|}, {\ex\verb|<problem>|}, {\ex\verb|<question>|}.\\
Axiom-like: {\ex\verb|<assumption>|}, {\ex\verb|<axiom>|}, {\ex\verb|<conjecture>|}, {\ex\verb|<heuristic>|}, {\ex\verb|<hypothesis>|}, {\ex\verb|<principle>|}.\\
Remark-like: {\ex\verb|<remark>|}, {\ex\verb|<convention>|}, {\ex\verb|<insight>|}, {\ex\verb|<note>|}, {\ex\verb|<observation>|}, {\ex\verb|<warning>|}.\\
Project-like: {\ex\verb|<project>|}, {\ex\verb|<activity>|}, {\ex\verb|<exploration>|}, {\ex\verb|<investigation>|}.\\

Other common blocks:\\
{\ex\verb|<definition>|}\\
{\ex\verb|<exercise>|}\\
{\ex\verb|<task>|}: a division of an exercise or project-like\\




\columnbreak

\hr\textbf{Examples}

Examples go here.





\end{multicols*}

\end{document}
